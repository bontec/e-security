
\documentclass[runningheads,a4paper]{article}

\usepackage{amssymb}
\setcounter{tocdepth}{3}
\usepackage{graphicx}
\usepackage{caption}
\usepackage{url}
\usepackage{amsmath}
\usepackage{multirow}
\usepackage{todonotes}
\usepackage{soul}

\begin{document}
\title{Qubes OS Security Analysis}

\author{\"Ulber Onur Akin, Charlotte Bonte}

\maketitle

\tableofcontents

\newpage
\section{Introduction}
Qubes OS is an open-source security-oriented operating system.  It is
based on Fedora, a Linux distribution, and Xen, a vitrual machine
monitor.  It has been developed by Invisible Things Lab founded by the
well-known computer security researcher, Joanna Rutkowska. It has been
initially released in 2012, while its latest version of R 4.0 was
released on March 28, 2018.

\begin{figure}
  \centering
  \includegraphics[width=1.0\textwidth]{qubes-arch-diagram.png}
  \caption{Qubes architecture overview
    Source:\protect\url{https://www.qubes-os.org/doc/architecture/}}
  \label{fig:qubesarch}
\end{figure}

Qubes OS is based on the assumption that it is impossible to create a
bug-free desktop environment and one critical bug can ruin the
security of the your system and have malicious software taking control
of your machine. Qubes OS plans to provide security by
compartmentalisation of the various actions you perform on your
computer into securely isolated compartments, called qubes.  This
separation assures that when one qube gets compromised, the others
remain unaffected, since they are isolated.  This technique is
referred to as security by compartmentalization or security by
isolation.  Qubes uses virtualization technology to isolate the
different applications from each other. Next to this approach,
sandboxing is used for different system-level components, such as
networking and storage subsystems again to avoid that a compromise in
one of these affects the integrity of your system. As such, Qubes OS
allows you to perform all your actions on one physical computer
without having to worry that one attack may compromise your whole
system. Qubes integrates all these secure containers into a unified
desktop environment. This provides a better user experience over
traditional methods of isolation. The different qubes are implemented
as lightweight virtual machines(VMs), that do not interfere with each
other.  Note that Qubes OS enables users to have copy-paste operations
and file sharing between the different security domains, but other
than these intrinsic mechanisms there is no information flow between
the different VMs, hence no unauthorized data flow can take place. An
overview of the Qubes architecture is shown in
Figure~\ref{fig:qubesarch}.

Not every app runs in its own qube for efficiency reasons. Instead
each qube represents a security domain. In the default settings three
qubes will be created: work, personal and unthrusted. All these
qubes/VMs are integrated in Qubes OS onto one common desktop
environment. This integration is provided by Application Viewer, which
meakes it appear that all applications run on a native desktop. The
security domain of an applications is indicated with a colored frame.
%In this report, we will
% asses the Qubes operating system (OS) based on chapter 7 and 8 of
% \cite{GollmannComputerSecurity}. As Qubes OS is linux based we expect
% to have similarities with chapter 7 ``Unix Security'' of
% \cite{GollmannComputerSecurity}.
\todo{check out https://www.qubes-os.org/doc/tools/ for experiments}
\todo{test qubes\_guid in dom0 terminal:  qubes\_guid} %-d domain_id [-c color] [-l label_index] [-i icon name, no suffix] [-v] [-q]}

\section{Principals} 

The principals are the active entities in a security policy, they are
the entities that can be granted or denied access. The principals of
Qubes OS are mainly the users or the security domains/qubes.\todo{also
different machines possible?} As Qubes OS was originally not designed
as a multi-user system, the users are local users who are administered
locally to the local computer and the operating system itself. This
user is responsible for his own security. Qubes OS is a platform that
offers the possibility to perform certain actions in a secure way by
their isolated set-up. However, you as a user are responsible for
using the system correctly such that the features they included to
provide security are actually used in the foreseen way. The different
security domains will correspond to different qubes/VMs that a user
can setup on his machine. Adding, removing and listing the different
VMs can be easily done in the Qubes WM Manager, with the add and
remove buttons or from command line in a console running in dom0,
using the commands
\begin{verbatim}
qvm-create
qvm-remove
qvm-ls
\end{verbatim}
For each of these VMs the user will be able to set properties like if
it is allowed to connect to the network, whether files can be copied
to it. \todo{find the way to do this,
https://www.qubes-os.org/news/2017/10/03/core3/}


\subsection{The different domains}
As said before, the whole system is build up from different
domains. There are some fixed domains like the core of the system
dom0, the network domain and the storage domain. Next to those there
are dedicated domains by default work, personal and unthrusted but the
user can create as many as he wants and also disposable domains which
are created to perform one specific task and deleted upon completion
of this task.

\subsubsection{Dom0}

Dom0 is the core of the system, which manages the virtual disks of the
other VMs, which are actually stored as files on the dom0
filesystem. The different virtual machines share the same root file
system in a read-only-mode. This allows centralized software
installation and updates. For the user's directory and the settings
for each VM seperate disk storage is used. 
%One can install software on
%a specific VM, by installing it as the non-root user or by installing
%it in the non-standard /rw hierarchy. 
Dom0  hosts the GUI(graphical user interface) domain and controls the
graphics device as well as input devices like the keyboard and the
mouse. It also
manages the other VMs via the Qubes VM manager. Hence dom0 functions as the thrusted computing base. If dom0
gets comprimised the whole system is corrupted and there is no
security left. The recomendations are that you never connect it to the
network itself and it is not used for web surfing or working with
files.

Since dom0 is the thrusted computing base, all aspects of the Qubes
system can be controlled using a terminal in the dom0 console. Opening
a console windom in dom0 can be done by:
\begin{itemize}
\item choose Terminal Emulator in the start menu
\item press Alt-F3, type xfce terminal and press enter twice
\item right-click on the desktop and select Open Terminal Here
\end{itemize}
Alternatively to the console you can work with the Qubes VM
manager. It supports most of the functionality provided through
command line. It is opened automaticly when Qubes OS starts and can
also be opened through Start $\rightarrow$ System Tools $\rightarrow$
Qubes manager.

The GUI domain runs the X server that is used to display the user desktop and
the window manager that allows to start and stop applications and
manipulate their windows. The Application Viewer provides the
integration of the different virtual machines by displaying the
content of the applications that are hosted in the
different AppVMs in one integrated desktop view for
the user while in fact all apllications are hosted in isolated AppVMs.


\subsubsection{Network domain}

Qubes OS makes use of the standard Xen networking, with however one
big change. This means it will consist of a pair of network devices,
the frontend will reside in the VMs that want to use the network, the
guest domain and the backend will reside in the backend
domain. Typically this backend, that contains the core networking
code, which includes network card drivers and various protocal stacks
runs in the kernel. In Qubes OS, this backend will not be the dom0
domain, but a driver domain\footnote{A driver domain is an
unpriviledged Xen domain that handles one specific piece of
hardware. The main idea is again that hardware drivers are
failure-prone and it is good for safety to isolate them from the rest
of the system.}, called the Network domain. The Intel VT-d technology
is used to safely grant the unprivileged the network domain access to
the networking hardware by safe device assigment. Network protocols
and routing might have exploitable bugs, by putting this functionality
in an unpriviledged domain, other domains will not be compromised even
if the attack succeeds. All sensitive applications are hosted in other
domains should use network protocols that are protected by
cryptography such as SSL, VPN or SSH.

Routed networking is used, as well as  network address tranlation(NAT)
at each network hop to eliminate layer 2 attacks coming from a
compromised VM. The default virtual network interface (vif) routing
script is replaced with the custom \textbf{vif-route-qubes} script
because the original contained some deficiencies. When the
Qubes-database is booted it assignes the IP address of eth0 interface
in AppVM and two IP addresses to be used as nameservers. These last
two are private addresses. Every time an network interface is
initiated the \textbf{/usr/lib/qubes/qubes\_setup\_dnat\_to\_ns} script
maps the two private IP addresses to real DNS servers by application
of the DNAT iptables rules\footnote{Iptables is a software firewall that
enforces network-level policies. This firewalling happens in a
seperated VM and not in the network VM, because the network VM is
considered easy-to-compromise and hence unthrusted.}. This networking
setup assures the AppVM networking configuration can remain unchanged
when the network configuration of the network driver domain
changes. In addition, there is no need for a DNS server in the network
driver domain and therefore there are no ports open to the VMs.
\todo{add VM routing table(short) and network driver domain routing
  table(longer)}

\subsubsection{Firewall domain}

In Qubes OS every domain connected to the network is connected with
the network domain through a firewall VM. This VM is used to enforce
network-level policies. By default Qubes OS generates one firewall VM
but as with the dedicated VMs users can create more firewall VMs if
required. The firewall rules of a specific VM are stored in an XML
file \texttt{/var/lib/qubes/appvms/<vm-name>/firewall.xml} in that VM's
directory in dom0 . Hence the rules can be adapted from the dom0
domain, either by selecting the specific VM in the Qubes manager and
using the firewall button or through the \texttt{qvm-firewall} command
in the terminal. Since Qubes version R4.0 some rules are no longer
accessible in the Qubes manager GUI, like for example ICMP and DNS,
however they can still be adapted with the terminal command. A last
important note is that the size of the \texttt{iptables} script is
limited to 3kB, when this limit is exceede the corresponding qube will
not start. Enforcing rules on the qube itself can circumvent this
limit.\todo{print screen of default rules through qubes manager or
  terminal}

\subsubsection{Storage domain}

The storage domain contains disk/USB/DVD drivers, stacks and
filesystem backends. Since Qubes OS is designed to optimize the disk
usage of the virtual machines, it has a complex file sharing mechanism
and structure on the backend side. Because more complicated structures
are often more error-prone and hence increase the attack surface, the
developers of Qubes OS decided to keep this code outside the dom0
domain. The filesystem sharing code, the backends, together with disk
drivers and stacks are therefore moved to an unprivileged VM which is
called the storge domain. Intel VT-d technology is used again, this
time to grant the network domain access to the disk controller. To
mitigate attacks originated from removable storage devices, USB flash
drivers or CD/DVD disks are also handled by the storage domain, which
has therefore access to USB and CD/DVD controllers.

To make this domain secure the Qubes architecture uses cryptography to
protect the filesystems of other domains from being comrpomised by the
storage domain in a meaningful way. Three important mechanisms are put
into place to make sure the storage domain can not be abused for
attacking the whole system. The storage devices of each VM are
encrypted with a key only known to the AppVM and dom0, the root
filesystem used by the AppVMs is located on a signed block device of
which only the UpdateVM and dom0 know the signing key and the boot
proces is based on a thrusted platform module(TPM) and is implemented
with the Intel TXT(thrusted execution technology). The signing of the
block containing the root file system makes sure that alteration of
this files can not go unnoticed, this block is not encrypted since
then all AppVms should have access to the decryption key and hence
when only one of the AppVMs and the storage domain are compromised by
the same attacker, this adversary be able to alther the root file
system. By signing the AppVMs only need the public verification key
and hence the above scenario is unapplicable. An attacker who
compromised the storage domain can alter the hypervisor of dom0 image
stored in the boot partition through his access to the disk
controller. When booting the system the compromised hypervisor will be
loaded and started, but when a thrusted boot scheme in this case based
on Intel TXT is used the TPM will not release the secret keys needed
to decrypt the files of the different VMs, so the system will not boot
properly and the adversary does not gain access to the rest of the
filesystems. Hence, the compromise of the storage domain can lead to a
denial-of-service attack but the attacker can never tamper with the
private data or root filesystem in a meaningful way.

\subsubsection{Dedicated domains}

In these domains the user can host user applications, browsing the
web, working on files, etc. These virtual machines are called the
AppVMs. In different AppVMs one can perform similar tasks, however
applications ran in different AppVMs will never influence each other. 
For example the same web browser can be
open in both the working and the unthrusted domain. In the working
domain this can be used for opening your work mail and in the
unthrusted domain, you can visit unknown websites. These browsers are
running on virtual machines and are therefore independent of each
other. This implies that if you are logged in into your work email in
the web browser of the work domain, you will not be logged in when you
open the web link of your work email in the unthrusted domain. Hence
your work email can not be influenced by malicious software appearing
in the unthrusted domain. These are the kind of setups that the Qubes
OS developers envisioned when developing Qubes OS.

AppVMs can be initialised with different system
distributions. As said before, all AppVMs
based on the same system distribution will share the same read-only
file system and seperate disk storage is only used for user's
directory and per-VM settings. The root file system will be located in
e.g. \texttt{/boot, /bin, /usr}, the private AppVM data in
e.g. \texttt{/home, /usr/local} and \texttt{/var}. 
Since this introduces complexity in the
filesystem, Qubes OS will again isolate this into the previously
mentioned storage doma which sandboxes
all disk and file system storage code, such that even if this gets
compromised, no harm can be done to the rest of the system.

\subsubsection{Disposable domains}

Disposable domains are created to perform one single task. Once this
task is completed, the domain will be gone as well. Upon clompleting
the action in a disposable domain, everything of that session will be
gone, opening later the same application in that disposable domain
will show nothing saved from the previous session. A disposable domain
can be opened from any domain by a right-click option. So after
downloading a file in the work domain or personal domain you can first
open it in a disposable domain, this avoids contamination of your
dedicated doamin. This mechanism of disposable domains is constructed
following the idea of sandboxes.


\section{Subjects}

As an OS from the UNIX-like OS family, in Qubes OS
subjects are processes. They can be implemented by commands on the
terminal. To see the activities of CPU we should apply the command of
“xentop”. To illustrate, from our computer we get such an output when
we use this command:
\todo{difference between real uid and effecctive uid? can you do dom0
  actions from terminal in fe personal?}\todo{no groups?}

\subsection{Login and passwords} 
\todo{TEST passwd in dom0 terminal to change password (look for
  password file, as in unix in /etc/passwd?) what does it mean to have
an empty string for the password in the password file, in unix this
means no password so free access,* for password means no login
possible as in unix? password file stores only hashes of
passwords?}\todo{probably no passwords in /etc/password but shadow
password file in /.secure/etc/passwd that can only be accessed by
root}

In Qubes OS, login is basically done
from the main domain called Dom0.  Since the Dom0 is more trusted than
the other domains it is authorized for login operations.  As a single
user OS, Qubes OS recognizes and authenticates user via password on
the Dom0 terminal.  Changing password is also possible.  If you use
the command of “passwd” command on Dom0 terminal, you can easily
change your password.  In Qubes, there is no a direct password
restriction.  You can choose any password you wish.

\subsection{Shadow password files} 

As it is stated in the Computer
Security book of Gollmann, if you store your passwords in an explicit
file, or in a VM in our Qubes OS case, it is possible for the hackers
to obtain our all passwords even if you hash them.  (As we know, from
the most commonly used passwords, like 123456, cracking the Hash codes
is not an extremely rare issue for hackers even if you choose a decent
password.)  This danger disables our gains from isolated environment.
To prevent this we should implement shadow paging or in other words we
should store this kind of data on shadowed VM’s or a subfile in a VM.
This is provided by XEN and hence Qubes OS.  If we refer to the
founder of Qubes OS, Joanna Rutkowska, she recommends to use KeePassX
and running it in an network-isolated dedicated AppVM.
\todo{uses password salting?, check /etc/shadow} 

\section{Objects}

The objects of access control are the resources, this includes files, directories,
memory devices and I/O devices. These resources are organised in a
three-structured system. \todo{how are the resources organised in
  qubes os? is it three-structured system} These objects always exist
in a specific domain and hence are accesible in that domain. 

However, the Qubes framework also has a need for communication between
different domains. For instance, when a user selects an application in
the desktop menu, it should be started in the chosen VM, hence dom0
should be able to start applications in an AppVM. Communication in the
other direction can also be interesting, for example to forward error
messages from a specific domain to dom0. To achieve this inter-VM
communication the Qubs qrexec framework wwas created, it has a
corresponding policy framework that controls the actions that can be
performed. Dom0 has a directory \texttt{/etc/qubes-rpc/policy} with
files, whose names describe the available RPC(remote procedure call)
actions. These files contain the access policy database. Some examples
of the RPC action in qubes are:
\begin{verbatim}
qubes.Filecopy
qubes.OpenInVM
qubes.ReceiveUpdates
qubes.VMShell
qubes.ClipboardPaste
\end{verbatim} 
The access policy is formulated in the following format
\begin{verbatim}
srcvm destvm (allow|deny|ask)
[,user=user\_to\_run\_as][,target=VM\_to\_redirect\_to]
\end{verbatim}
where \texttt{srcvm} and \texttt{destvm} can be specified by name or
by reserved keywords, such as dom0, \$anyvm or \$dispvm. \$anyvm
stands for all existing VMs except dom0. Service calls
from dom0 are always allowed and \$dispvm can not be a source vm since
it stands for a new VM created for this particular request, so this
can only be placed as destvm.

Whenever a RPC request is received, the corresponding policy file will
be checked. The first line of the policy file that has matching srcvm
and destvm of the request determines whether this RPC request is
allowed, what user account this program should run the target VM under
and what VM to redirect the execution to. Creating and maintaining
these policy files is again the responsibility of the user. When the
policy file does not exist upon a RPC request, the user is prompted to
create one manually. If after the prompting there is still no policy
file in place then the action is denied. Hence by default the action
is denied unless the user explicitely allows this request by creating
a line in the policy file. Next to this policy check,
the actual RPC action should exist on the target VM, this means
\texttt{/etc/qubes-rpc/RPCrequest} must exist containing the file name
of the program that will be invoke upon this request or being the
program itself. When the program itself is placed here it must have
the execuatble permission set. \todo{this is set as chmod -x in linux,
  so depending on os that the VM uses it internally uses these
  security features.}

As example we show the RPC policy for copying a file as it was created
in dom0.
\todo{experiment create policy file for FileCopy as in
  https://www.qubes-os.org/doc/rpc-policy/ and test whether it is true
  that the first applicable rule grants or denies access.}

\subsection{Copy and paste between domains} 

Qubes OS supports a secure
copy and paste operation between domains.  The copy-paste operation
happens through a clipboard.  To copy text from domain A to domain B,
you start by copying some text in an applications window of domain A,
it will copy this into the domain's local clipboard.  Then by using
the key combination Ctrl-Shift-C you let Qubes know that you want to
select this domain's clipboard for global copy between domains.  The
next step is to select the destination application in domain B and use
the key combination Ctrl-Shirt-V.  This will make the global clipboard
avaliable to apps running in domain B.  This step ensures that only
the applications of domain B will get access tot the clipboard copied
from domain A.  The last action to take is to use the destination
application specific key combination for pasting the clipboard and
then you copied the text from domain A to domain B.

 This 'secure' copying is called secure since it does not allow VMs
outside the security domain B to steal the content of the clipboard.
However, it does not prevent writing from a less trusted to a more
trusted domain.  It leaves this responsibility to the user itself.
The policy can be configured, in the sense that you can set the policy
to never allow pastes into certain domains.  This is done by altering
the policy in /etc/qubes-rpc/policy/qubes. ClipboardPaste with the
command \$anyvm vault deny \$anyvm \$anyvm ask (try this out in VM)
\todo{Unfortunately, the default Terminal application is the GNOME
Terminal (v3.22.2) which has already mapped Ctrl-Shift-v. Therefore,
if you want to paste into a Terminal, you have to start at least a
second application in the target VM. Before pasting, you have to
switch to this second application, invoke the non-mapped Ctrl-Shift-v
and then switch back to Terminal and paste with Ctrl-Shift-v. This a
really silly usability bug. }

\subsubsection{Copying from and to dom0} 

Copying from Dom0 to another
VM is done with the command
\begin{verbatim}
qvm-copy-to-vm <dest-vm><file>
\end{verbatim}.
The file then arrives in the destination VM in the
~/QubesIncoming/dom0/ directory.

The developers do not encourage copying files from VMs to Dom0, since
Dom0 acts as a trusted terminal hance no user applications should be
ran there.  In addition, one should take great care when copying into
Dom0 since Dom0 controls the whole system and all security is lost if
it gets compromised.  It can however be done using the following
command in Dom0's console.  
\begin{verbatim}
qvm-run --pass-io <src-vm> 'cat /path/to/file_in_src_domain' > /path/to/file_name_in_dom0 
\end{verbatim} 
 A similar command can also be used to copy from Dom0 to other VMs if
you do not want to use the previous command qvm-copy-to-vm for some
reason.  
\begin{verbatim}
cat /path/to/file_in_dom0 | qvm-run --pass-io <dest-vm> 'cat > /path/to/file_name_in_appvm'
\end{verbatim}

\subsection{Logging}
Operating systems are developed by human beings which means they can
have some flaws inside them. In order to detect this flaws and keep
track of security breaches operating systems do logging a lot. The
Qubes OS VM log files and the log files of the AppVMs can be found in
the directory \texttt{/var/log/qubes}. The log file per VM are
seperated in a file for graphical information, one for sound
information, one for inter VM communication and one for qubesdb
information, the names of these log files are respectively
\texttt{guid.<vmname>.log, pacat.<vmname>.log, qrexec.<vmname>.log}
and \texttt{qubesdb.<vmname>.log}.

\section{Management fucntionalities}
\subsection{Multi-users}
As mentioned before, Qubes OS is originally not designed as a
multi-user system.  A multi-user system was considered but the
developers found to many potential attack vecctors, they believe the
building bloks of their system are not fit to ensure security in a
multi-user environment, so it can not be implemmented securely at the
moment. Important to mention is that in any case, the user who
controls Dom0 always controls the whole system. The main focus of
Qubes OS is protecting the user from various treats, rather than
proctecting the system from the user.

However, in order to become an operating system that can be used in corperate environments or large organisations, there must be a mechanism in Qubes OS to make it remotely manageable by entities such as IT departments.
Therefore, the developers created several new
management functionalities in Qubes OS R4.0. They can be basically examined under two
main groups called Admin API and Core Stackv3. Next to these management functionalities there are several other improvments like
fully virtualized VMs, multiple and more flexible disposable VM templates, a more expressive and user-friendly Qubes RPC policy system, a more powerful VM volume manager that enables us to keep VMs on external drives easier, enhanced template VM security via splitted packages and network interface, more secure backups with scrypt for stronger key derivation and enforced encryption and command-line tools with new options. In our report, we will mainly focus on the remote managing functions, namely, Admin API and Core Stack. In addition, we will briefly mention the fully virtualized VMs, since they offer a partial solution to the recently discovered meltdown and spectre attacks.

\subsection{Admin API}
Admin API aims to enable users to manage the domains without direct access to dom0. Briefly, it allows the users to remotely manage multiple Qubes machines. The biggest motivation to implement this functionality was targetting the enterprise customers and corporations. The developers believe that enabling IT administrators to remotely manage multiple Qubes machines would be very useful. The Admin API would also allow the creation of a multi-user system on a Qubes machine, where each user can have different, as well as the same if desired, set of secure domains/environments. It is implemented via a dedicated semi-trusted VM. In addition, it is allowed to ask Dom0 for certain actions by using a characteristically designed qrexec-based protocol. The units by which the API might be utilised are basically Qubes OS Manager, CLI tools, remote management tools, custom tools etc. Some basic facilities of Admin API can be stated as remote manageability, GUI domain (a trusted GUI subsystem and the admin stack combined in the same AppVM), safe third party templates and so on. For the ones who have concerns about potential security gaps and vulnerabilities, the project’s and whole Qubes OS’ co-founder Joanna Rutkowska asserted the Admin API does not have an access to the lowest level of the system known as Dom0 as an endorsement of the robustness of the security. She stated that the Admin API could even be limited such that it can still allow an administrator to configure VMs, but without being able to read the system user’s data. 

\subsection{Core stack}
By its official developers’ definition, Qubes Core Stack is the core component of Qubes OS which connects all the other components together. Apart from the core (main) component, these components mainly are VM-located core agents, VM-customizations, Qubes GUI virtualization, GUI domain customization, the AdminVM distribution, the XEN hypervisor, multiple Qubes Apps, several ready-to-use templates, salt stacks integration etc.
By doing this, it enables the users and admins to contact with each other and configure the system as well. 

\subsection{Fully Virtualized VMs (HVMs)}
By default, VMs of Qubes OS were using para-virtualization technology, chosen before for being easier to implement and having higher performance. However, Qubes OS, from 4.0 version on, implements fully virtualized or hardware assisted VMs to offer better protection against Meltdown attacks. According to the Qubes team, full virtualization might also provide protection against at least one of the Spectre vulnerabilities. Because in this case the attacker needs to exploit the hypervisor, which is obviously more difficult than exploiting a full OS. Potential attacks would also be solely read-only and the hackers would not be able to install a backdoor in the system through either Meltdown or Spectre. 


%\section{Access control}

\bibliographystyle{alpha}

\bibliography{cryptobib/abbrev1,cryptobib/crypto,references}

\end{document}
