\documentclass[runningheads,a4paper]{article}

\usepackage{amssymb}
\setcounter{tocdepth}{3}
\usepackage{graphicx}
\usepackage{caption}
\usepackage{url}
\usepackage{amsmath}
\usepackage{multirow}
\usepackage{todonotes}

\begin{document}
\title{Qubes OS Security Analysis}

\author{\"Ulber Onur Akin, Charlotte Bonte}

\maketitle

\tableofcontents
\section{Introduction}

Qubes OS is an open-source security-oriented operating system.  It is
based on Fedora, a Linux distribution, and Xen, a vitrual machine
monitor.  It has been developed by Invisible Things Lab founded by the
well-known computer security researcher, Joanna Rutkowska. It has been
initially released in 2012, while its latest version of R 4.0 was
released on March 28, 2018.

Qubes OS is based on the assumption that it is impossibleto create a
bug-free desktop environment and one critical bug can ruin the
security of the your system and have malicious software taking control
of your machine. Qubes OS plans to provide security by
compartmentalisation of the various actions 
you perform on your computer into securely isolated compartments,
called qubes.  This separation assures that when one qube gets
compromised, the others remain unaffected, since they are isolated.
This technique is referred to as security by compartmentalization or
security by isolation.  Qubes integrates all these secure containers
into a unified desktop environment.  As such, Qubes OS allows you to
perform all your actions on one physical computer without having to
worry that one attack may compromise your whole system.  This provides
a better user experience over traditional methods of isolation.  The
different qubes are implemented as lightweight Xen virtual machines,
that do not interfere with each other.  Note that Qubes OS enables
users to have copy-paste operations and file sharing between the
different security domains, but other than these intrinsic mechanisms
there is no information flow between the different VMs, hence no
unauthorized data flow can take place.  

Not every app runs in its own qube for efficiency reasons. Instead
each qube represents a security domain. In the default settings three
qubes will be created: work, personal and unthrusted. All these
qubes/VMs are integrated in Qubes OS onto one common desktop
environment. This integration is provided by Application Viewer, which
meakes it appear that all applications run on a native desktop. The
security domain of an applications is indicated with a colored frame.
%In this report, we will
% asses the Qubes operating system (OS) based on chapter 7 and 8 of
% \cite{GollmannComputerSecurity}. As Qubes OS is linux based we expect
% to have similarities with chapter 7 ``Unix Security'' of
% \cite{GollmannComputerSecurity}.

\section{Principals} 

The principals are the active entities in a security policy, they are
the entities that can be granted or denied access. The principals of
Qubes OS are mainly the users or the security domains/qubes.\todo{also
different machines possible?} As Qubes OS was originally not designed
as a multi-user system, the users are local users who are administered
locally to the local computer and the operating system itself. This
user is responsible for his own security. Qubes OS is a platform that
offers the possibility to perform certain actions in a secure way by
their isolated set-up. However, you as a user are responsible for
using the system correctly such that the features they included to
provide security are actually used in the foreseen way. The different
security domains will correspond to different qubes/VMs that a user
can setup on his machine.  Adding, removing and listing the different
VMs can be easily done in the Qubes WM Manager, with the add and
remove buttons or from command line in a console running in dom0,
using the commands
\begin{verbatim}
qvm-create
qvm-remove
qvm-ls
\end{verbatim}
For each of these VMs the user will be able to set properties like if
it is allowed to connect to the network, whether files can be copied
to it. \todo{find the way to do this,
https://www.qubes-os.org/news/2017/10/03/core3/}


\subsection{The different security domains}
As said before, the whole system is build up from different
domains. There are some fixed domains like the core of the system
dom0, the network domain and the storage domain. Next to those there
are dedicated domains by default work, personal and unthrusted but the
user can create as many as he wants and also disposable domains which
are created to perform one specific task and deleted upon completion
of this task.

\subsubsection{Dom0}

Dom0 is the core of the system, which manages the virtual disks of the
other VMs, which are actually stored as files on the dom0
filesystem. The different virtual machines share the same root file
system in a read-only-mode. This allows centralized software
installation and updates. For the user's directory and the settings
for each VM seperate disk storage is used. One can install software on
a specific VM, by installing it as the non-root user or by installing
it in the non-standard /rw hierarchy. It will be used for displaying,
hence as graphical user interface, managing keyboard and mouse input,
managing the other VMs via the Qubes VM manager and system
network. Hence dom0 functions as the thrusted computing base. If dom0
gets comprimised the whole system is corrupted and there is no
security left. The recomendations are that you never connect it to the
network itself and it is not used for web surfing or working with
files.

\subsubsection{Network domain}

\subsubsection{Storage domain}

\subsubsection{Dedicated domains}

In these domains the user can perform several actions, browsing the web, working on files, etc. For example the same web browser can be open in both the working and the unthrusted domain. In the working domain this can be used for opening your work mail and in the unthrusted domain, you can visit unknown websites. These browsers are running on virtual machines and are therefore independent of each other. This implies that if you are logged in into your work email in the web browser of the work domain, you will not be logged in when you open the web link of your work email in the unthrusted domain. Hence your work email can not be influenced by malicious software appearing in the unthrusted domain. These are the kind of setups that the Qubes OS developers envisioned when developing Qubes OS.

\subsubsection{Disposable domains}

Disposable domains are created to perform one single task. Once this task is completed, the domain will be gone as well. Upon clompleting the action in a disposable domain, everything of that session will be gone, opening later the same application in that disposable domain will show nothing saved from the previous session. A disposable domain can be opened from any domain by a right-click option. So after downloading a file in the work domain or personal domain you can first open it in a disposable domain, this avoids contamination of your dedicated doamin. 
This mechanism of disposable domains is constructed following the idea of sandboxes.


\section{Subjects}

As an OS from the UNIX-like OS family, in Qubes OS
subjects are They. processes can be implemented by commands on the
terminal. To see the activities of CPU we should apply the command of
“xentop”. To illustrate, from our computer we get such an output when
we use this command:

\subsection{Login and passwords} 
\todo{TEST passwd in dom0 terminal to change password (look for password file)}

In Qubes OS, login is basically done
from the main domain called Dom0.  Since the Dom0 is more trusted than
the other domains it is authorized for login operations.  As a single
user OS, Qubes OS recognizes and authenticates user via password on
the Dom0 terminal.  Changing password is also possible.  If you use
the command of “passwd” command on Dom0 terminal, you can easily
change your password.  In Qubes, there is no a direct password
restriction.  You can choose any password you wish.

\subsection{Shadow password files} 

As it is stated in the Computer
Security book of Gollmann, if you store your passwords in an explicit
file, or in a VM in our Qubes OS case, it is possible for the hackers
to obtain our all passwords even if you hash them.  (As we know, from
the most commonly used passwords, like 123456, cracking the Hash codes
is not an extremely rare issue for hackers even if you choose a decent
password.)  This danger disables our gains from isolated environment.
To prevent this we should implement shadow paging or in other words we
should store this kind of data on shadowed VM’s or a subfile in a VM.
This is provided by XEN and hence Qubes OS.  If we refer to the
founder of Qubes OS, Joanna Rutkowska, she recommends to use KeePassX
and running it in an network-isolated dedicated AppVM.

\section{Objects}
\subsection{Copy and paste between domains} 

Qubes OS supports a secure
copy and paste operation between domains.  The copy-paste operation
happens through a clipboard.  To copy text from domain A to domain B,
you start by copying some text in an applications window of domain A,
it will copy this into the domain's local clipboard.  Then by using
the key combination Ctrl-Shift-C you let Qubes know that you want to
select this domain's clipboard for global copy between domains.  The
next step is to select the destination application in domain B and use
the key combination Ctrl-Shirt-V.  This will make the global clipboard
avaliable to apps running in domain B.  This step ensures that only
the applications of domain B will get access tot the clipboard copied
from domain A.  The last action to take is to use the destination
application specific key combination for pasting the clipboard and
then you copied the text from domain A to domain B.\\ This 'secure'
copying is called secure since it does not allow VMs outside the
security domain B to steal the content of the clipboard.  However, it
does not prevent writing from a less trusted to a more trusted domain.
It leaves this responsibility to the user itself.  The policy can be
configured, in the sense that you can set the policy to never allow
pastes into certain domains.  This is done by altering the policy in
/etc/qubes-rpc/policy/qubes.ClipboardPaste with the command \$anyvm
vault deny \$anyvm \$anyvm ask (try this out in VM)

\subsubsection{Copying from and to dom0} 

Copying from Dom0 to another
VM is done with the command
\begin{verbatim}
qvm-copy-to-vm <dest-vm><file>
\end{verbatim}.
The file then arrives in the destination VM in the
~/QubesIncoming/dom0/ directory.

The developers do not encourage copying files from VMs to Dom0, since
Dom0 acts as a trusted terminal hance no user applications should be
ran there.  In addition, one should take great care when copying into
Dom0 since Dom0 controls the whole system and all security is lost if
it gets compromised.  It can however be done using the following
command in Dom0's console.  
\begin{verbatim}
qvm-run --pass-io <src-vm> 'cat /path/to/file_in_src_domain' > /path/to/file_name_in_dom0 
\end{verbatim} 
 A similar command can also be used to copy from Dom0 to other VMs if
you do not want to use the previous command qvm-copy-to-vm for some
reason.  
\begin{verbatim}
cat /path/to/file_in_dom0 | qvm-run --pass-io <dest-vm> 'cat > /path/to/file_name_in_appvm'
\end{verbatim}

\section{The architecture}
As mentioned before, Qubes OS is originally not designed as a
multi-user system.  A multi-user system was considered but the
developers found to many potential attack vecctors, they believe the
building bloks of their system are not fit to ensure security in a
multi-user environment, so it can not be implemmented securely at the
moment. Important to mention is that in any case, the user who
controls Dom0 always controls the whole system. The main focus of
Qubes OS is protecting the user from various treats, rather than
proctecting the system from the user.

However, from Qubes 4 on, the developers created several new
management functionalities. They can be basically examined under two
main groups called Admin API and Core Stack.
% As it is strictly mentioned on its official website’s FAQ part, Qubes
% is originally, regarding its security-obsessive nature, has not been a
% multi-user system and it has no claim or pretention about supporting
% multi-user environment, currently. Contrarily, since Qubes believes
% that it is very difficult to ensure security in multi-user OS since
% one controls Dom0 has a control on the overall system too. While the
% developers were creating the Qubes OS, they used to consider some ways
% of creating unprivileged multi-user accounts in Dom0 such that user
% Alice didn’t have access to user Bob’s AppVMs. But they came up with
% many threats as potential attack vectorsfrom such unprivileged user
% account to system admin (root). Hence they decided not to implement
% multi-user system.  Nevertheless, from Qubes 4 on, there are several
% new management functionalities. They can be basically examined under
% two main groups called Admin API and Core Stack.  %Note that, the next
% two implemenatations of “Admin API” and “Core Stack” are innovations
% peculiar to Qubes 4, hence the previous versions of Qubes did not have
% those. To be more updated, we would rather focus on the latest version
% i.e Qubes 4. Also we found those functionalities significant and
% noteworthy.

\subsection{Admin API}

\subsection{Core stack}


\section{Access control}

\bibliographystyle{alpha}
\bibliography{cryptobib/abbrev1,cryptobib/crypto,references}

\end{document}
